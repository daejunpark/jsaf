%%%%%%%%%%%%%%%%%%%%%%%%%%%%%%%%%%%%%%%%%%%%%%%%%%%%%%%%%%%%%%%%%%%%%%%%%%%%%%%%
%   Copyright 2012, KAIST.
%   All rights reserved.
%
%   Use is subject to license terms.
%
%   This distribution may include materials developed by third parties.
%%%%%%%%%%%%%%%%%%%%%%%%%%%%%%%%%%%%%%%%%%%%%%%%%%%%%%%%%%%%%%%%%%%%%%%%%%%%%%%%

\documentclass[a4paper]{article}

\usepackage{amsmath, amsfonts, amssymb, amsthm}
\usepackage{stmaryrd}
\usepackage{color}

\newcommand{\Program}{\emph{Program}}
\newcommand{\SourceElement}{\emph{SourceElement}}
\newcommand{\Statement}{\emph{Statement}}
\newcommand{\VariableDeclaration}{\emph{VariableDeclaration}}
\newcommand{\FunctionDeclaration}{\emph{FunctionDeclaration}}
\newcommand{\ModuleDeclaration}{\emph{ModuleDeclaration}}
\newcommand{\ModuleBody}{\emph{ModuleBody}}
\newcommand{\ModuleElement}{\emph{ModuleElement}}
\newcommand{\ExportDeclaration}{\emph{ExportDeclaration}}
\newcommand{\ExportSpecifierSet}{\emph{ExportSpecifierSet}}
\newcommand{\ExportSpecifier}{\emph{ExportSpecifier}}
\newcommand{\Path}{\emph{Path}}
\newcommand{\ModuleSpecifier}{\emph{ModuleSpecifier}}
\newcommand{\ImportDeclaration}{\emph{ImportDeclaration}}
\newcommand{\ImportClause}{\emph{ImportClause}}
\newcommand{\ImportSpecifierSet}{\emph{ImportSpecifierSet}}
\newcommand{\ImportSpecifier}{\emph{ImportSpecifier}}
\newcommand{\ModuleImport}{\emph{ModuleImport}}
\newcommand{\FunctionBody}{\emph{FunctionBody}}
\newcommand{\Expression}{\emph{Expression}}
\newcommand{\StringLiteral}{\emph{StringLiteral}}
\newcommand{\Identifier}{\emph{Identifier}}
\newcommand{\Env}{\emph{Env}}
\newcommand{\var}{{\tt{var}}}
\newcommand{\module}{{\tt{module}}}
\newcommand{\local}{{\tt{local}}}
\newcommand{\export}{{\tt{export}}}
\newcommand{\code}[1]{\textcolor{blue}{\tt #1}}
\newcommand{\If}{\mbox{if}}
\newcommand{\Otherwise}{\mbox{otherwise}}
\newcommand{\Where}{\mbox{where}}
\newcommand{\lookup}{\emph{lookup}}
\newcommand{\FunName}{\emph{FunName}}
\newcommand{\VarName}{\emph{VarName}}
\newcommand{\PriVarName}{\emph{PriVarName}}
\newcommand{\PubVarName}{\emph{PubVarName}}
\newcommand{\ModName}{\emph{ModName}}
\newcommand{\UpdateName}{\emph{UpdateName}}
\newcommand{\HoistedName}{\emph{HoistedName}}
\newcommand{\FunDecl}{\emph{FunDecl}}
\newcommand{\VarDecl}{\emph{VarDecl}}
\newcommand{\Exports}{\emph{Exports}}
\newcommand{\PriVarDecl}{\emph{PriVarDecl}}
\newcommand{\PubVarDecl}{\emph{PubVarDecl}}
\newcommand{\ModDecl}{\emph{ModDecl}}
\newcommand{\UpdateFun}{\emph{UpdateFun}}
\newcommand{\UpdateArgs}{\emph{UpdateFun}}
\newcommand{\QualName}{\emph{QualName}}
\newcommand{\QualNameEnv}{\emph{QualNameEnv}}
\newcommand{\SetPrototype}{\emph{SetPrototype}}
\newcommand{\N}{\mathbb{N}}
\newcommand{\caret}{\mathbin{\char`\^}}
\newcommand{\mydot}{.\,}
\newcommand{\TODO}{\mbox{\textcolor{red}{TODO:}\ }}

\title{Rewriting JavaScript Module \\
       \fbox{\small\it Working Draft}}
\author{Junhee Cho, Sukyoung Ryu}

\begin{document}

\maketitle

\section{Environment}

\TODO The construction of environments for \code{export *} is omitted for now.

\begin{eqnarray*}
x & \in & \Identifier \\
\phi & ::= & x(\code{\mydot}x)* \\
\varphi_i & ::= & \phi\code{\mydot(}x\code{)} \\
\varphi_e & ::= & \phi\mydot x \\
\varphi & ::= & \varphi_i \mid \varphi_e \\
\overline{\varphi} & ::= & \phi\code{\mydot*} \\
\tau & = & \var \mid \module \\
\varsigma & = & \epsilon \mid \local \mid \export\ \varphi_e \\
\rho & = & \bot \mid \tau \varphi \\
\overline{\rho} & = & \bot \mid \top \\
\Sigma & = & \{(\varphi, \rho \varsigma), \cdots\} \cup \{(\overline{\varphi}, \overline{\rho}), \cdots\} \\
\Sigma^* & = & \epsilon \mid \Sigma^* \Sigma \mid \Sigma^* x
\end{eqnarray*}

\begin{equation*}
\begin{split}
&\Env_m \llbracket \code{var}\ x_1 (\ \code{=}\ e_1)\ \code{,}\ \cdots\ \code{,}\ x_n (\ \code{=}\ e_n)\ \code{;} \rrbracket (\Sigma; \phi) = \\
&\qquad\left\{\begin{array}{ll}
\{(\phi.(x), \var\ \phi.(x)\ \export\ \phi.x), (\phi.x, \var\ \phi.x\ \epsilon) \mid x \in \{x_1, \cdots, x_n\}\} & \If\ \phi = \epsilon \\
\{(\phi.(x), \var\ \phi.(x)\ \epsilon) \mid x \in \{x_1, \cdots, x_n\}\} & \Otherwise
\end{array}\right. \\
&\Env_m \llbracket \code{module}\ M\ \code{\{} s_1 \cdots s_n \code{\}} \rrbracket (\Sigma; \phi) = \\
&\qquad\{(\phi.(M), \module\ \phi.(M)\ \epsilon)\} \cup \Env_m \llbracket s_1 \cdots s_n \rrbracket (\Sigma; \phi.M) \\
&\Env_m \llbracket \code{export}\ \emph{ess}_1\code{,}\ \cdots\code{,}\ \emph{ess}_n \code{;} \rrbracket (\Sigma; \phi) = \\
&\qquad\bigcup_{1 \le i \le n} \Env_m \llbracket \code{export}\ \emph{ess}_i \rrbracket (\Sigma; \phi) \\
&\Env_m \llbracket \code{export}\ \code{\{}\emph{es}_1\code{,}\ \cdots\code{,}\ \emph{es}_n\code{\}}\ (\code{from}\ p) \rrbracket (\Sigma; \phi) = \\
&\qquad\bigcup_{1 \le i \le n} \Env_m \llbracket \code{export}\ \code{\{}es_i\code{\}}\ (\code{from}\ p) \rrbracket (\Sigma; \phi) \\
&\Env_m \llbracket \code{export}\ \code{\{}x\code{\}}\ (\code{from}\ p) \rrbracket (\Sigma; \phi) = \\
&\qquad\Env_m \llbracket \code{export}\ \code{\{}x\code{:}\ x\code{\}}\ (\code{from}\ p) \rrbracket (\Sigma; \phi) \\
&\Env_m \llbracket \code{export}\ \code{\{}x\code{:}\ p\code{\}}\ \code{from}\ p_0 \rrbracket (\Sigma; \phi) = \\
&\qquad\Env_m \llbracket \code{export}\ \code{\{}x\code{:}\ p_0\code{.}p\code{\}} \rrbracket (\Sigma; \phi) \\
&\Env_m \llbracket \code{export}\ \code{\{}x\code{:}\ x_0\code{\}} \rrbracket (\Sigma; \phi) = \\
&\qquad\left\{\begin{array}{ll}
\{(\phi.(x_0), \bot\ \export\ \phi.x), (\phi.x, \bot\ \epsilon), (\phi.x, \var\ \phi.x\ \epsilon)\} & \If\ (\phi.(x_0), \var\ \phi'.(x')\ \varsigma) \in \Sigma \\
\{(\phi.(x_0), \bot\ \export\ \phi.x), (\phi.x, \bot\ \epsilon), (\phi.x, \var\ \phi'.x'\ \epsilon)\} & \If\ (\phi.(x_0), \var\ \phi'.x'\ \varsigma) \in \Sigma \\
\{(\phi.(x_0), \bot\ \export\ \phi.x), (\phi.x, \bot\ \epsilon)\} & \Otherwise
\end{array}\right. \\
&\Env_m \llbracket \code{export}\ \code{\{}x\code{:}\ p.x_0\code{\}} \rrbracket (\Sigma; \phi) = \\
&\qquad\left\{\begin{array}{ll}
\{(\phi.x, \bot), (\phi.x, \var\ \varphi\ \epsilon)\} & \If\ \lookup \llbracket p.x_0 \rrbracket (\Sigma; \phi) = \var\ \varphi \\
\{(\phi.x, \bot)\} & \Otherwise
\end{array}\right. \\
&\Env_m \llbracket \code{export}\ x\ (\code{from}\ p) \rrbracket (\Sigma; \phi) = \\
&\qquad\Env_m \llbracket \code{export}\ \code{\{}x\code{\}}\ (\code{from}\ p) \rrbracket (\Sigma; \phi) = \\
&\Env_m \llbracket \code{export}\ \code{var}\ x_1(\ \code{=}\ e_1)\ \code{,}\ \cdots\ \code{,}\ x_n(\ \code{=}\ e_n)\ \code{;} \rrbracket (\Sigma; \phi) = \\
&\qquad\{(\phi.(x), \var\ \phi.(x)\ \export\ \phi.x), (\phi.x, \var\ \phi.x\ \epsilon) \mid x \in \{x_1, \cdots, x_n\}\} \\
&\Env_m \llbracket \code{export}\ \code{function}\ f\code{(}x_1,\ \cdots,\ x_n\code{)}\ \code{\{}\ s_1,\ \cdots,\ s_m\ \code{\}} \rrbracket (\Sigma; \phi) = \\
&\qquad\Env_m \llbracket \code{export}\ \code{var}\ f \rrbracket (\Sigma; \phi) \\
&\Env_m \llbracket \code{export}\ \code{get}\ f\code{()}\ \code{\{}\ s_1,\ \cdots,\ s_m\ \code{\}} \rrbracket (\Sigma; \phi) = \\
&\qquad\Env_m \llbracket \code{export}\ \code{var}\ f \rrbracket (\Sigma; \phi) \\
&\Env_m \llbracket \code{export}\ \code{set}\ f\code{(}x_1,\ \cdots,\ x_n\code{)}\ \code{\{}\ s_1,\ \cdots,\ s_m\ \code{\}} \rrbracket (\Sigma; \phi) = \\
&\qquad\Env_m \llbracket \code{export}\ \code{var}\ f \rrbracket (\Sigma; \phi) \\
\end{split}
\end{equation*}
\begin{equation*}
\begin{split}
&\Env_m \llbracket \code{import}\ \emph{ic}_1\code{,}\ \cdots\code{,}\ \emph{ic}_n \rrbracket (\Sigma; \phi) = \\
&\qquad\bigcup_{1 \le i \le n} \Env_m \llbracket \code{import}\ \emph{ic}_i \rrbracket (\Sigma; \phi) \\
&\Env_m \llbracket \code{import}\ p\ \code{as}\ M \rrbracket (\Sigma; \phi) = \\
&\qquad\left\{\begin{array}{ll}
\{(\phi.(M), \bot\ \epsilon), (\phi.(M), \module\ \varphi\ \epsilon)\} & \If\ \lookup \llbracket p \rrbracket (\Sigma; \phi) = \module\ \varphi \\
\{(\phi.(M), \bot\ \epsilon)\} & \Otherwise
\end{array}\right. \\
&\Env_m \llbracket \code{import}\ \code{\{} \emph{is}_1\code{,}\ \cdots\code{,}\ \emph{is}_n \code{\}}\ \code{from}\ p \rrbracket (\Sigma; \phi) = \\
&\qquad\bigcup_{1 \le i \le n} \Env_m \llbracket \code{import}\ \code{\{} \emph{is}_1 \code{\}}\ \code{from}\ p \rrbracket (\Sigma; \phi) \\
&\Env_m \llbracket \code{import}\ \code{\{} x \code{\}}\ \code{from}\ p \rrbracket (\Sigma; \phi) = \\
&\qquad\Env_m \llbracket \code{import}\ \code{\{} x\code{:}\ x \code{\}}\ \code{from}\ p \rrbracket (\Sigma; \phi) \\
&\Env_m \llbracket \code{import}\ \code{\{} x\code{:}\ x' \code{\}}\ \code{from}\ p \rrbracket (\Sigma; \phi) = \\
&\qquad\left\{\begin{array}{ll}
\{(\phi.(x'), \bot\ \epsilon), (\phi.(x'), \var\ \varphi\ \epsilon)\} & \If\ \lookup \llbracket p.x \rrbracket (\Sigma; \phi) = \tau\ \varphi \\
\{(\phi.(x'), \bot\ \epsilon)\} & \Otherwise
\end{array}\right. \\
&\Env_m \llbracket \code{import}\ x\ \code{from}\ p \rrbracket (\Sigma; \phi) = \\
&\qquad\Env_m \llbracket \code{import}\ \code{\{} x \code{\}}\ \code{from}\ p \rrbracket (\Sigma; \phi) \\
\end{split}
\end{equation*}

\begin{eqnarray*}
&\lookup \llbracket M.\phi \rrbracket (\Sigma; \phi_0) & = \left\{\begin{array}{ll}
\lookup' \llbracket \phi \rrbracket (\Sigma; \phi'.M') & \If\ (\phi_0.(M), \module\ \phi'.(M')\ \varsigma) \in \Sigma \\
\lookup_\bot \llbracket M.\phi \rrbracket (\Sigma; \phi_0) & \Otherwise
\end{array}\right. \\
&\lookup \llbracket x \rrbracket (\Sigma; \phi_0) & = \left\{\begin{array}{ll}
\tau\ \varphi & \If\ (\phi_0.(x), \tau\ \varphi\ \varsigma) \in \Sigma \\
\lookup_\bot \llbracket x \rrbracket (\Sigma; \phi_0) & \Otherwise
\end{array}\right. \\
&\lookup_\bot \llbracket \phi \rrbracket & = \left\{\begin{array}{ll}
\lookup \llbracket \phi \rrbracket (\Sigma; \phi'_0) & \If\ \phi_0 = \phi'_0.M \\
\bot & \Otherwise
\end{array}\right. \\
&\lookup' \llbracket M.\phi \rrbracket (\Sigma; \phi_0) & = \left\{\begin{array}{ll}
\lookup' \llbracket \phi \rrbracket (\Sigma; \phi'.M') & \If\ (\phi_0.M, \module\ \phi'.(M')\ \varsigma) \in \Sigma \\
\bot & \Otherwise
\end{array}\right. \\
&\lookup' \llbracket x \rrbracket (\Sigma; \phi_0) & = \left\{\begin{array}{ll}
\tau\ \varsigma & \If\ (\phi_0.x, \tau\ \varphi\ \varsigma) \in \Sigma \\
\bot & \Otherwise
\end{array}\right. \\
\end{eqnarray*}

\begin{equation*}
\begin{split}
&\Env \llbracket p \rrbracket = \bigcup_{i \ge 0} \Env_m^i \llbracket p \rrbracket (\emptyset; \epsilon) \\
\Where\ & \Env_m^0 \llbracket p \rrbracket (\Sigma; \phi) = \Sigma \\
& \Env_m^i \llbracket p \rrbracket (\Sigma; \phi) = \Env_m \llbracket p \rrbracket (\Env_m^{i-1} \llbracket p \rrbracket (\Sigma; \phi); \phi)\ \mbox{for}\ i \ge 0 \\
\end{split}
\end{equation*}

\section{Rewriting Rule}
\subsection{Program}
Rewriting of a program is to collect module declarations, import declarations, and the others in the order, then to rewrite each of them. Hoisting function declarations and variable declarations are omitted because it will be done by every JavaScript interpreter. Rewriting module declarations consists of two parts; to initialize all the module instance objects with function declarations and variable declarations in the modules, then to interpret the module bodies and freezing the module instance objects.
\begin{equation*}
\frac{(\Sigma, \phi), p \rightarrow_m p_m \qquad (\Sigma, \phi), p \rightarrow_u p_u \qquad (\Sigma, \phi), p \rightarrow_s p_s}{(\Sigma, \phi), p \rightarrow p_m\ p_u\ p_s}
\end{equation*}

\subsection{Module Declaration}
\subsubsection{Names of Function Declarations}
\FunName\ is a helper function that collects the names of functions declared in a module.
\begin{equation*}
\FunName \llbracket s_1 \cdots s_n \rrbracket (\Sigma, \phi) = \bigcup_{1 \le i \le n} \FunName \llbracket s_i \rrbracket (\Sigma, \phi)
\end{equation*}
\begin{equation*}
\FunName \llbracket (\code{export})\ \code{function}\ f\code{(}x_1\code{,}\ \cdots\code{,}\ x_n\code{)}\ \code{\{}\ s_1\ \cdots\ s_m\ \code{\}} \rrbracket (\Sigma, \phi) = \{ f \}
\end{equation*}

\subsubsection{Names of Variables in Module}
\VarName, \PriVarName, \PubVarName\ are helper functions that collect the names of variables, unexported variables, and exported variables in a module, respectively.
\begin{equation*}
\VarName \llbracket s_1 \cdots s_n \rrbracket (\Sigma, \phi) = \{ x \mid (\phi.(x), \var\ \varphi\ \varsigma) \in \Sigma \} \setminus \FunName \llbracket s_1 \cdots s_n \rrbracket (\Sigma, \phi)
\end{equation*}
\begin{equation*}
\PriVarName \llbracket s_1 \cdots s_n \rrbracket (\Sigma, \phi) = \{ x \mid (\phi.(x), \var\ \varphi\ \varsigma) \in \Sigma \wedge (\phi.x, \var\ \varphi\ \varsigma) \notin \Sigma\} \setminus \FunName \llbracket s_1 \cdots s_n \rrbracket (\Sigma, \phi)
\end{equation*}
\begin{equation*}
\PubVarName \llbracket s_1 \cdots s_n \rrbracket (\Sigma, \phi) = \{ x \mid (\phi.x, \var\ \varphi\ \varsigma) \in \Sigma \} \setminus \FunName \llbracket s_1 \cdots s_n \rrbracket (\Sigma, \phi)
\end{equation*}

\subsubsection{Names of Submodules in Module}
\ModName\ is a helper function that collects the names of submodule declared in a module.
\begin{equation*}
\ModName (\Sigma, \phi) = \{ M \mid (\phi.(M), \module\ \varphi\ \varsigma) \in \Sigma \}
\end{equation*}

\subsubsection{Fresh Name of Update Function for Module}
\UpdateName\ is a helper function that finds a unused name in a module for temporary helper function for the corresponding module instance object.
\begin{equation*}
\begin{split}
\UpdateName (\Sigma, \phi) & = \code{update}k \\
                           & \Where\ k = \min\{ n \mid n \in \N \wedge (\phi.(\code{update}n), \tau\ \varphi\ \varsigma) \in \Sigma \}
\end{split}
\end{equation*}

\subsubsection{Function Declaration}
\FunDecl\ is a helper function that collects the function declarations declared in a module.
\begin{equation*}
\FunDecl \llbracket s_1 \cdots s_n \rrbracket (\Sigma, \phi) = \bigcup_{1 \le i \le n} \FunDecl \llbracket s_1 \rrbracket (\Sigma, \phi)
\end{equation*}
\begin{equation*}
\begin{split}
&\FunDecl \llbracket (\code{export})\ \code{function}\ f\code{(}x_1\code{,}\ \cdots\code{,}\ x_n\code{)}\ \code{\{}\ s_1\ \cdots\ s_n\ \code{\}} \rrbracket (\Sigma, \phi) = \\
&\qquad \code{function}\ f\code{(}x_1\code{,}\ \cdots\code{,}\ x_n\code{)}\ \code{\{}\ s_1 \cdots s_n\ \code{\}}
\end{split}
\end{equation*}

\subsubsection{Variable Declaration}
\VarDecl\ is a helper function that declares all the variables in a module.
\begin{equation*}
\VarDecl \llbracket s_1 \cdots s_n \rrbracket (\Sigma, \phi) = \{ \code{var}\ x \mid x \in \VarName \llbracket s_1 \cdots s_n \rrbracket (\Sigma, \phi) \}
\end{equation*}

\subsubsection{Exports}
\Exports\ is a helper function that sets the getters for exported names in a module.
\begin{equation*}
\begin{split}
&\Exports (\Sigma, \phi) = \\
&\qquad \{ \code{Object.defineProperty(this,}\ \code{``}x\code{",}\ \code{\{get:}\ \code{function()}\ \code{\{}\ \code{return}\ x\code{;}\ \code{\}\})} \mid (\phi.x, \tau\ \varphi\ \varsigma) \in \Sigma \}
\end{split}
\end{equation*}

\subsubsection{Update Function}
\UpdateFun\ is a helper function that generates a helper function in module instance object, of which body is the module body except the function, variable, and submodule declarations.
\begin{equation*}
\UpdateFun \llbracket s_1\ \cdots\ s_n \rrbracket (\Sigma, \phi) = \bigcup_{1 \le i \le n} \UpdateFun \llbracket s_i \rrbracket (\Sigma, \phi)
\end{equation*}
%\begin{equation*}
%\UpdateFun \llbracket (\code{export})\ \code{function}\ f\code{(}x_1\code{,}\ \cdots\code{,}\ x_n\code{)}\ \code{\{}\ s_1\ \cdots\ s_m\ \code{\}} \rrbracket (\Sigma, \phi) = \epsilon
%\end{equation*}
%\begin{equation*}
%\UpdateFun \llbracket (\code{export})\ \code{module}\ M\ \code{\{}\ s_1\ \cdots\ s_n\ \code{\}} \rrbracket (\Sigma, \phi) = \epsilon
%\end{equation*}
\begin{equation*}
\UpdateFun \llbracket (\code{export})\ \code{var}\ x_1(\code{=}\ e_1)\code{,}\ \cdots\code{,}\ x_n(\code{=}\ e_n) \rrbracket (\Sigma, \phi) = \{ x_i\ \code{=}\ e_i \mid 1 \le i \le n \wedge \exists e_i \}
\end{equation*}
\begin{equation*}
\begin{split}
&\UpdateFun \llbracket \code{export}\ \code{get}\ f\code{()}\ \code{\{}\ s_1 \cdots s_n\ \code{\}} \rrbracket (\Sigma, \phi) = \\
&\qquad \code{if}\ \code{(typeof}\ \code{Object.getOwnPropertyDescriptor(this,}\ \code{``}f\code{")}\ \code{===}\ \code{``undefined")} \\
&\qquad\qquad \code{Object.defineProperty(this,}\ \code{``}f\code{",}\ \code{\{} \\
&\qquad\qquad\qquad \code{get:}\ \code{function()}\ \code{\{}\ s_1 \cdots s_n\ \code{\}\ \})} \\
&\qquad \code{else} \\
&\qquad\qquad \code{Object.defineProperty(this,}\ \code{``}f\code{",}\ \code{\{} \\
&\qquad\qquad\qquad \code{get:}\ \code{function()}\ \code{\{}\ s_1 \cdots s_n\ \code{\},} \\
&\qquad\qquad\qquad \code{set:}\ \code{Object.getOwnPropertyDescriptor(this,}\ \code{``}f\code{").set\ \})}
\end{split}
\end{equation*}
\begin{equation*}
\begin{split}
&\UpdateFun \llbracket \code{export}\ \code{set}\ f\code{()}\ \code{\{}\ s_1 \cdots s_n\ \code{\}} \rrbracket (\Sigma, \phi) = \\
&\qquad \code{if}\ \code{(typeof}\ \code{Object.getOwnPropertyDescriptor(this,}\ \code{``}f\code{")}\ \code{===}\ \code{``undefined")} \\
&\qquad\qquad \code{Object.defineProperty(this,}\ \code{``}f\code{",}\ \code{\{} \\
&\qquad\qquad\qquad \code{set:}\ \code{function()}\ \code{\{}\ s_1 \cdots s_n\ \code{\}\ \})} \\
&\qquad \code{else} \\
&\qquad\qquad \code{Object.defineProperty(this,}\ \code{``}f\code{",}\ \code{\{} \\
&\qquad\qquad\qquad \code{get:}\ \code{Object.getOwnPropertyDescriptor(this,}\ \code{``}f\code{").get,} \\
&\qquad\qquad\qquad \code{set:}\ \code{function()}\ \code{\{}\ s_1 \cdots s_n\ \code{\}\ \})}
\end{split}
\end{equation*}
%\begin{equation*}
%\UpdateFun \llbracket \code{import}\ \emph{ic}_1\code{,}\ \cdots\code{,}\ \emph{ic}_n \rrbracket (\Sigma, \phi) = \bigcup_{1 \le i \le n} \UpdateFun \llbracket \code{import}\ \emph{ic}_i \rrbracket (\Sigma, \phi)
%\end{equation*}
%\begin{equation*}
%\UpdateFun \llbracket \code{import}\ x\ \code{as}\ x' \rrbracket (\Sigma, \phi) = x'\ \code{=}\ \code{arguments.}x
%\end{equation*}
%\begin{equation*}
%\UpdateFun \llbracket \code{import}\ p.x\ \code{as}\ x' \rrbracket (\Sigma, \phi) = x'\ \code{=}\ \code{arguments.}x
%\end{equation*}
%\begin{equation*}
%\UpdateFun \llbracket \code{import}\ \code{\{}\emph{is}_1\code{,}\ \cdots\code{,}\ \emph{is}_n\code{\}}\ \code{from}\ p \rrbracket (\Sigma, \phi) = \bigcup_{1 \le i \le n} \UpdateFun \llbracket \code{import}\ \code{\{}\emph{is}_i\code{\}}\ \code{from}\ p \rrbracket (\Sigma, \phi)
%\end{equation*}
%\begin{equation*}
%\UpdateFun \llbracket \code{import}\ x\ \code{from}\ p \rrbracket (\Sigma, \phi) = \UpdateFun \llbracket \code{import}\ \code{\{}x\code{\}}\ \code{from}\ p \rrbracket (\Sigma, \phi)
%\end{equation*}
%\begin{equation*}
%\UpdateFun \llbracket \code{import}\ \code{\{}x\code{\}}\ \code{from}\ p \rrbracket (\Sigma, \phi) = \UpdateFun \llbracket \code{import}\ \code{\{}x\code{:}\ x\code{\}}\ \code{from}\ p \rrbracket (\Sigma, \phi)
%\end{equation*}
%\begin{equation*}
%\UpdateFun \llbracket \code{import}\ \code{\{}x\code{:}\ x'\code{\}}\ \code{from}\ p \rrbracket (\Sigma, \phi) = x'\ \code{=}\ \code{arguments.}x
%\end{equation*}
\begin{equation*}
\UpdateFun \llbracket s \rrbracket (\Sigma, \phi) = s \qquad\If\ s \in \Statement
\end{equation*}

\subsubsection{Substituting Qualified Name in Expression}
$\QualName_e$\ is a helper function that substitutes imported identifiers in an expression to the corresponding exported qualified names.
\begin{eqnarray*}
\QualName_e \llbracket \code{this} \rrbracket (\Gamma) & = & \code{this} \\
\QualName_e \llbracket x \rrbracket (\Gamma) & = &
\left\{\begin{array}{ll}
\Gamma[x] & \If\ x \in \Gamma \\
x & \Otherwise
\end{array}\right. \\
\QualName_e \llbracket \emph{literal} \rrbracket (\Gamma) & = & \emph{literal} \\
\QualName_e \llbracket \code{[}(e_1)\code{,}\ \cdots\code{,}\ (e_n)\code{]} \rrbracket (\Gamma) & = & \code{[}(\QualName \llbracket e_1 \rrbracket (\Gamma))\code{,}\ \cdots\code{,}\ (\QualName \llbracket e_n \rrbracket (\Gamma))\code{]} \\
\QualName_e \llbracket \code{\{}e_1\code{,}\ \cdots\code{,}\ e_n(\code{,})\code{\}} \rrbracket (\Gamma) & = & \code{\{}\QualName \llbracket e_1 \rrbracket (\Gamma)\code{,}\ \cdots\code{,}\ \QualName \llbracket e_n \rrbracket (\Gamma)(\code{,})\code{\}} \\
\QualName_e \llbracket \code{(}e\code{)} \rrbracket (\Gamma) & = & \code{(}\QualName \llbracket e \rrbracket (\Gamma)\code{)}
\end{eqnarray*}

\begin{equation*}
\QualName_e \llbracket x\code{:}\ e \rrbracket (\Gamma) = x\code{:}\ \QualName \llbracket e \rrbracket (\Gamma)
\end{equation*}
\begin{equation*}
\begin{split}
&\QualName_e \llbracket \code{get}\ f\code{()}\ \code{\{}\ s_1\ \cdots\ s_n\ \code{\}} \rrbracket (\Gamma) = \code{get}\ f\code{()}\ \code{\{}\ \QualName_s \llbracket s_1 \rrbracket (\Gamma')\ \cdots\ \QualName_s \llbracket s_n \rrbracket (\Gamma')\ \code{\}} \\
\Where\ & \Gamma' = \Gamma \setminus \{f, \code{arguments}\} \setminus \HoistedName \llbracket s_1 \cdots s_n \rrbracket
\end{split}
\end{equation*}
\begin{equation*}
\begin{split}
&\QualName_e \llbracket \code{set}\ f\code{(}x\code{)}\ \code{\{}\ s_1\ \cdots\ s_n\ \code{\}} \rrbracket (\Gamma) = \code{set}\ f\code{(}x\code{)}\ \code{\{}\ \QualName_s \llbracket s_1 \rrbracket (\Gamma')\ \cdots\ \QualName_s \llbracket s_n \rrbracket (\Gamma')\ \code{\}} \\
\Where\ & \Gamma' = \Gamma \setminus \{f, x, \code{arguments}\} \setminus \HoistedName \llbracket s_1 \cdots s_n \rrbracket
\end{split}
\end{equation*}

\begin{equation*}
\begin{split}
&\QualName_e \llbracket \code{function}\ f\code{(}x_1\code{,}\ \cdots\code{,}\ x_n\code{)}\ \code{\{}\ s_1\ \cdots\ s_n\ \code{\}} \rrbracket (\Gamma) = \\
&\qquad\code{function}\ f\code{(}x_1\code{,}\ \cdots\code{,}\ x_n\code{)}\ \code{\{}\ \QualName_s \llbracket s_1 \rrbracket (\Gamma \setminus \{f, x_1, \cdots, x_n\})\ \cdots\ \QualName_s \llbracket s_n \rrbracket (\Gamma')\ \code{\}} \\
&\Where\ \Gamma' = \Gamma \setminus \{f, x_1, \cdots, x_n, \code{arguments}\} \setminus \HoistedName \llbracket s_1 \cdots s_n \rrbracket
\end{split}
\end{equation*}
\begin{equation*}
\begin{split}
&\QualName_e \llbracket \code{function}\ \code{(}x_1\code{,}\ \cdots\code{,}\ x_n\code{)}\ \code{\{}\ s_1\ \cdots\ s_n\ \code{\}} \rrbracket (\Gamma) = \\
&\qquad\code{function}\ \code{(}x_1\code{,}\ \cdots\code{,}\ x_n\code{)}\ \code{\{}\ \QualName_s \llbracket s_1 \rrbracket (\Gamma \setminus \{x_1, \cdots, x_n\})\ \cdots\ \QualName_s \llbracket s_n \rrbracket (\Gamma')\ \code{\}} \\
&\Where\ \Gamma' = \Gamma \setminus \{x_1, \cdots, x_n, \code{arguments}\} \setminus \HoistedName \llbracket s_1 \cdots s_n \rrbracket
\end{split}
\end{equation*}

\begin{eqnarray*}
\QualName_e \llbracket e_1\code{[}e_2\code{]} \rrbracket (\Gamma) & = & \QualName_e \llbracket e_1 \rrbracket (\Gamma)\code{[}\QualName_e \llbracket e_2 \rrbracket (\Gamma)\code{]} \\
\QualName_e \llbracket e\code{.}x \rrbracket (\Gamma) & = & \QualName_e \llbracket e_1 \rrbracket (\Gamma)\code{.}x \\
\QualName_e \llbracket \code{new}\ e\code{(}e_1\code{,}\ \cdots\code{,}\ e_n\code{)} \rrbracket & = & \code{new}\ \QualName_e \llbracket e \rrbracket (\Gamma)\code{(}\QualName_e \llbracket e_1 \rrbracket (\Gamma)\code{,}\ \cdots\code{,}\ \QualName_e \llbracket e_n \rrbracket (\Gamma)\code{)} \\
\QualName_e \llbracket \code{new}\ e \rrbracket (\Gamma) & = & \code{new}\ \QualName_e \llbracket e \rrbracket (\Gamma) \\
\QualName_e \llbracket e\code{(}e_1\code{,}\ \cdots\code{,}\ e_n\code{)} \rrbracket & = & \QualName_e \llbracket e \rrbracket (\Gamma)\code{(}\QualName_e \llbracket e_1 \rrbracket (\Gamma)\code{,}\ \cdots\code{,}\ \QualName_e \llbracket e_n \rrbracket (\Gamma)\code{)}
\end{eqnarray*}

\begin{equation*}
\QualName_e \llbracket e[\code{++}\mid\code{--}] \rrbracket (\Gamma) = \QualName_e \llbracket e \rrbracket (\Gamma)[\code{++}\mid\code{--}]
\end{equation*}
\begin{equation*}
\begin{split}
&\QualName_e \llbracket \emph{op}\ e \rrbracket (\Gamma) = \emph{op}\ \QualName_e \llbracket e \rrbracket (\Gamma) \\
\Where\ &\emph{op} \in \{\code{delete}, \code{void}, \code{typeof}, \code{++}, \code{--}, \code{+}, \code{-}, \code{\sim}, \code{!}\}
\end{split}
\end{equation*}
\begin{equation*}
\begin{split}
&\QualName_e \llbracket e_1\ \emph{op}\ e_2 \rrbracket (\Gamma) = \QualName \llbracket e_1 \rrbracket (\Gamma)\ \emph{op}\ \QualName_e \llbracket e_2 \rrbracket (\Gamma) \qquad\Where\\
&\emph{op} \in \left\{\begin{array}{l}
\code{*}, \code{/}, \code{\%}, \code{+}, \code{-}, \code{<<}, \code{>>}, \code{>>>}, \code{<}, \code{>}, \code{<=}, \code{>=}, \code{instanceof}, \code{in}, \code{==}, \code{!=}, \code{===}, \code{!==}, \\
\code{\&\&}, \code{\mid\mid}, \code{\&}, \code{\caret}, \code{\mid}, \code{=}, \code{*=}, \code{/=}, \code{\%=}, \code{+=}, \code{-=}, \code{<<=}, \code{>>=}, \code{>>>=}, \code{\&=}, \code{\caret=}, \code{!=}, \code{,}
\end{array}\right\}
\end{split}
\end{equation*}

\subsubsection{Substituting Qualified Name in Statement}
$\QualName_s$\ is a helper function that substitutes imported identifiers in a statement to the corresponding exported qualified names.
\begin{equation*}
\QualName_s \llbracket \code{\{}\ s_1\ \cdots\ s_n\ \code{\}} \rrbracket (\Gamma) = \code{\{}\ \QualName_s \llbracket s_1 \rrbracket (\Gamma)\ \cdots\ \QualName_s \llbracket s_n \rrbracket (\Gamma)\ \code{\}}
\end{equation*}
\begin{equation*}
\QualName_s \llbracket \code{var}\ x_1\ (\code{=}\ e_1)\code{,}\ \cdots\code{,}\ x_n\ (\code{=}\ e_n) \rrbracket (\Gamma) = \code{var}\ x_1(\code{=}\ \QualName_e \llbracket e_1 \rrbracket (\Gamma))\code{,}\ \cdots\code{,}\ x_n(\code{=}\ \QualName_e \llbracket e_n \rrbracket (\Gamma))
\end{equation*}
\begin{equation*}
\QualName_s \llbracket \code{;} \rrbracket (\Gamma) = \code{;}
\end{equation*}
\begin{equation*}
\QualName_s \llbracket e \rrbracket (\Gamma) = \QualName_e \llbracket e \rrbracket (\Gamma)
\end{equation*}
\begin{equation*}
\QualName_s \llbracket \code{if}\ \code{(}e\code{)}\ s_1\ (\code{else}\ s_2) \rrbracket (\Gamma) = \code{if}\ \code{(}\QualName_e \llbracket e \rrbracket (\Gamma)\code{)}\ \QualName_s \llbracket s_1 \rrbracket (\Gamma)\ (\code{else}\ \QualName_s \llbracket s_2 \rrbracket (\Gamma))
\end{equation*}
\begin{equation*}
\QualName_s \llbracket \code{do}\ s\ \code{while}\ \code{(}e\code{);} \rrbracket (\Gamma) = \code{do}\ \QualName_s \llbracket s \rrbracket (\Gamma)\ \code{while}\ \code{(}\QualName_e \llbracket e \rrbracket (\Gamma)\code{);}
\end{equation*}
\begin{equation*}
\QualName_s \llbracket \code{while}\ \code{(}e\code{)}\ s \rrbracket (\Gamma) = \code{while}\ \code{(}\QualName_e \llbracket e \rrbracket (\Gamma)\code{)}\ \QualName_s \llbracket s \rrbracket (\Gamma)
\end{equation*}
\begin{equation*}
\begin{split}
&\QualName_s \llbracket \code{for}\ \code{(}e_1\code{;}\ e_2\code{;}\ e_3)\ s \rrbracket (\Gamma) = \\
&\qquad\code{for}\ \code{(}\QualName_e \llbracket e_1 \rrbracket (\Gamma)\code{;}\ \QualName_e \llbracket e_2 \rrbracket (\Gamma)\code{;}\ \QualName_e \llbracket e_3 \rrbracket (\Gamma)\code{)}\ \QualName_s \llbracket s \rrbracket (\Gamma)
\end{split}
\end{equation*}
\begin{equation*}
\begin{split}
&\QualName_s \llbracket \code{for}\ \code{(var}\ x_1(\code{=}\ e_1)\code{,}\ \cdots\code{,}\ x_n(\code{=}\ e_n)\code{;}\ e_{n+1}\code{;}\ e_{n+2}\code{)}\ s \rrbracket (\Gamma) \\
&\qquad\code{for}\ \code{(var}\ x_1(\code{=}\ \QualName_e \llbracket e_1 \rrbracket (\Gamma))\code{,}\ \cdots\code{,}\ x_n(\code{=}\ \QualName_e \llbracket e_n \rrbracket (\Gamma))\code{;}\ \\
&\qquad\qquad\QualName_e \llbracket e_{n+1} \rrbracket (\Gamma)\code{;}\ \\
&\qquad\qquad\QualName_e \llbracket e_{n+2} \rrbracket (\Gamma)\code{)}\ \QualName_s \llbracket s \rrbracket (\Gamma)
\end{split}
\end{equation*}
\begin{equation*}
\begin{split}
&\QualName_s \llbracket \code{for}\ \code{(}e_1\ \code{in}\ e_2)\ s \rrbracket (\Gamma) = \\
&\qquad\code{for}\ \code{(}\QualName_e \llbracket e_1 \rrbracket (\Gamma)\ \code{in}\ \QualName_e \llbracket e_2 \rrbracket (\Gamma)\code{)}\ \QualName_s \llbracket s \rrbracket (\Gamma)
\end{split}
\end{equation*}
\begin{equation*}
\begin{split}
&\QualName_s \llbracket \code{for}\ \code{(var}\ x_1(\code{=}\ e_1)\code{,}\ \cdots\code{,}\ x_n(\code{=}\ e_n)\ \code{in}\ e_{n+1}\code{)}\ s \rrbracket (\Gamma) \\
&\qquad\code{for}\ \code{(var}\ x_1(\code{=}\ \QualName_e \llbracket e_1 \rrbracket (\Gamma))\code{,}\ \cdots\code{,}\ x_n(\code{=}\ \QualName_e \llbracket e_n \rrbracket (\Gamma))\ \code{in}\ \QualName_e \llbracket e_{n+1} \rrbracket (\Gamma)\code{)}\ \\
&\qquad\qquad\QualName_s \llbracket s \rrbracket (\Gamma)
\end{split}
\end{equation*}
\begin{equation*}
\QualName_s \llbracket \code{continue}\ (x)\code{;} \rrbracket (\Gamma) = \code{continue}\ (x)\code{;}
\end{equation*}
\begin{equation*}
\QualName_s \llbracket \code{break}\ (x)\code{;} \rrbracket (\Gamma) = \code{break}\ (x)\code{;}
\end{equation*}
\begin{equation*}
\QualName_s \llbracket \code{return}\ (e)\code{;} \rrbracket (\Gamma) = \code{return}\ (\QualName_e \llbracket e \rrbracket (\Gamma))\code{;}
\end{equation*}
\begin{equation*}
\QualName_s \llbracket \code{with}\ \code{(}e\code{)}\ s \rrbracket (\Gamma) = \code{with}\ \code{(}\QualName_e \llbracket e \rrbracket (\Gamma)\code{)}\ \QualName_s \llbracket s \rrbracket (\Gamma)
\end{equation*}
\begin{equation*}
\QualName_s \llbracket \code{switch}\ \code{(}e\code{)}\ \code{\{}\ c_1\ \cdots\ c_n\ \code{\}} \rrbracket (\Gamma) = \code{switch}\ \code{(}\QualName_e \llbracket e \rrbracket (\Gamma)\code{)}\ \code{\{}\ \QualName_s \llbracket c_1 \rrbracket (\Gamma)\ \cdots\ \QualName_s \llbracket c_n \rrbracket (\Gamma)\ \code{\}}
\end{equation*}
\begin{equation*}
\QualName_s \llbracket \code{case}\ e\code{:}\ s_1\ \cdots\ s_n \rrbracket (\Sigma) = \code{case}\ e\code{:}\ \QualName_s \llbracket s_1 \rrbracket (\Sigma)\ \cdots\ \QualName_s \llbracket s_n \rrbracket (\Sigma)
\end{equation*}
\begin{equation*}
\QualName_s \llbracket \code{default:}\ s_1\ \cdots\ s_n \rrbracket (\Sigma) = \code{default:}\ \QualName_s \llbracket s_1 \rrbracket (\Sigma)\ \cdots\ \QualName_s \llbracket s_n \rrbracket (\Sigma)
\end{equation*}
\begin{equation*}
\QualName_s \llbracket x\code{:}\ s \rrbracket (\Sigma) = x\code{:}\ \QualName_s \llbracket s \rrbracket (\Sigma \setminus \{x\})
\end{equation*}
\begin{equation*}
\QualName_s \llbracket \code{throw}\ e \rrbracket (\Sigma) = \code{throw}\ \QualName_e \llbracket e \rrbracket (\Sigma)
\end{equation*}
\begin{equation*}
\begin{split}
&\QualName_s \llbracket \code{try}\ s_1\ (\code{catch}\ \code{(}x\code{)}\ s_2)\ (\code{finally}\ s_3) \rrbracket (\Sigma) = \\
&\qquad\qquad\code{try}\ \QualName_s \llbracket s_1 \rrbracket (\Sigma)\ (\code{catch}\ \code{(}x\code{)}\ \QualName_s \llbracket s_2 \rrbracket (\Sigma \setminus \{x\}))\ (\code{finally}\ \QualName_s \llbracket s_3 \rrbracket (\Sigma))
\end{split}
\end{equation*}
\begin{equation*}
\QualName_s \llbracket \code{debugger;} \rrbracket = \code{debugger;}
\end{equation*}

\subsubsection{Mapping to Qualified Name}
\QualNameEnv\ is a helper function that mapping from imported identifiers to the corresponding exported qualified names.
\begin{equation*}
\begin{split}
&\QualNameEnv (\Sigma, \phi) = \\
&\qquad \left\{\begin{array}{ll}
\{ x \mapsto \phi'.x' \mid (\phi.(x), \tau\ \phi'.x'\ \varsigma) \in \Sigma \} & \If\ \phi = \epsilon \\
\{ x \mapsto \phi'.x' \mid (\phi.(x), \tau\ \phi'.x'\ \varsigma) \in \Sigma \} \cup \QualNameEnv (\Sigma, \phi') & \If\ \phi = \phi'.M
\end{array}\right.
\end{split}
\end{equation*}

\subsubsection{Module Declaration}
For each module declaration, a function is introduced as a constructor that constructs the module instance object.
The function scope of the constructor is used as the module scope.
For each exported name, the module instance object has a getter for the property.
For mutually recursive imports, the constructor declares the functions and the variables in the module and the remaining parts are stored in a temporary helper function.
\begin{equation*}
\frac{(\Sigma, \phi), s_i \rightarrow_m s'_i \qquad\forall 1 \le i \le n}{(\Sigma, \phi), s_1 \cdots s_n \rightarrow_m s'_1 \cdots s'_n}
\end{equation*}
\begin{equation*}
(\Sigma, \phi), \code{module}\ M\ \code{\{}\ s_1 \cdots s_n\ \code{\}} \rightarrow_m \ModDecl \llbracket \code{module}\ M\ \code{\{}\ s_1 \cdots s_n\ \code{\}} \rrbracket (\Sigma, \phi)
\end{equation*}
\begin{equation*}
\begin{split}
&\ModDecl \llbracket \code{module}\ M\ \code{\{}\ s_1 \cdots s_n\ \code{\}} \rrbracket (\Sigma, \phi) = \\
&\qquad \code{var}\ M\ \code{=}\ \code{new}\ \SetPrototype (f, p)\ \code{()} \\
\Where\ & f = \code{function()}\ \code{\{} \\
&\qquad \QualName_e \llbracket \FunDecl \llbracket s_1 \cdots s_n \rrbracket (\Sigma, \phi) \rrbracket (\QualNameEnv(\Sigma, \phi)) \\
&\qquad \QualName_e \llbracket \VarDecl \llbracket s_1 \cdots s_n \rrbracket (\Sigma, \phi) \rrbracket (\QualNameEnv(\Sigma, \phi)) \\
&\qquad \QualName_e \llbracket \ModDecl \llbracket s_1 \cdots s_n \rrbracket (\Sigma, \phi) \rrbracket (\QualNameEnv(\Sigma, \phi)) \\
&\qquad \QualName_e \llbracket \Exports (\Sigma, \phi) \rrbracket (\QualNameEnv(\Sigma, \phi)) \\
&\qquad \code{this.}\UpdateName (\Sigma, \phi)\ \code{=}\ \code{function(arguments)}\ \code{\{}\ \\
&\qquad\qquad \QualName_s \llbracket \UpdateFun \llbracket s_1 \cdots s_n \rrbracket (\Sigma, \phi) \rrbracket (\QualNameEnv(\Sigma, \phi))\ \\
&\qquad \code{\}} \\
&\code{\}} \\
\mbox{and}\ & p = \left\{\begin{array}{ll}
\code{Object.prototype} & \If\ \phi = \epsilon \\
\phi & \Otherwise
\end{array}\right.
\end{split}
\end{equation*}
\begin{equation*}
\SetPrototype (f, p) = \code{function(f,}\ \code{ p)}\ \code{\{}\ \code{f.prototype}\ \code{=}\ \code{p}\code{;}\ \code{return}\ \code{f;}\ \code{\}}\ \code{(}f\code{,}\ p\code{)}
\end{equation*}

\subsection{Module Update}
Now, all the function and the variables in the module instance objects are declared.
In the order, the temporary helper functions are called, initiating the module instance objects.
Then, the helper functions are deleted and the module instance objects are freezed.
\begin{equation*}
\frac{\{ i_1, \cdots, i_k \} = \{ i \mid s_i \in \ModuleDeclaration \}}{(\Sigma, \phi), s_1 \cdots s_n \rightarrow_u s_{i_1} \cdots s_{i_k}}
\end{equation*}
\begin{equation*}
\frac{
\begin{array}{c}
(\Sigma, \phi), s_1 \cdots s_n \rightarrow_u s' \\
\begin{array}{l}
u = \code{\{} \\
\qquad \phi\code{.}M\code{.}\UpdateFun (\Sigma, \phi)\code{()} \\
\qquad \code{delete}\ \phi\code{.}M\code{.}\UpdateFun (\Sigma, \phi) \\
\qquad \code{Object.freeze(}\phi\code{.}M\code{)} \\
\qquad s' \\
\code{\}}
\end{array}
\end{array}
}{(\Sigma, \phi), \code{module}\ M\ \code{\{}\ s_1\ \cdots\ s_n\ \code{\}} \rightarrow_u u}
\end{equation*}

\subsection{Source Element}
In the rest of program, the imported names are renamed with the corresponding exported names.
\begin{equation*}
\frac{\{i_1, \cdots, i_k\} = \{i \mid 1 \le i \le n \wedge s_i \notin \ModuleDeclaration \cup \ImportDeclaration\}}{(\Sigma, \phi), s_1 \cdots s_n \rightarrow \QualName_s \llbracket s_{i_1} \cdots s_{i_k} \rrbracket (\QualNameEnv(\Sigma, \phi))}
\end{equation*}

\subsection{TODO}
\TODO How can I deal with the shadowing of \code{arguments}? \\
\TODO \code{Object.seal} between instantiation and initialization \\
\TODO Store the helper functions for initialization in the global context

\end{document}
